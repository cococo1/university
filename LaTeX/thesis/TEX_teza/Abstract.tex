\section*{Abstract}

Thesis \textbf{Object Clustering Using NAO Robot} presented by Maxim Chetrușca as a Bachelor project was developed at the Czech Technical University in Prague, is written in English and contains 90 pages, 26 figures, 7 tables, 13 listings and 20 references. The thesis consists of introduction, four chapters, conclusions and appendices. Appendices contain additional 6 figures, one table and 7 listings.

This thesis is dedicated to the study of robotics, Machine Learning, human-robotics interaction and image processing. The purpose of this thesis is to elaborate a system on a robot which would enable him to group a set of objects into multiple groups, independently of objects and number of groups.

Robotics is becoming an important branch of technology, not only for research but also as the next trend in commercial products. Most of the tasks performed by robots are related to interaction with objects and human-robotic interaction. The first implies handling the vision task of identifying the objects in space and grasping them. The second implies speech recognition and text-to-speech functionality. The vision task is accomplished using image processing. The positions of the objects relative to robot are computed to make a sense of space for the robot. This thesis describes a model of interaction in which a robot has to sort a set of objects. The task is accomplished using NAO humanoid robot, by means of image processing using OpenCV and K-means clustering algorithm. 

The thesis contains descriptions of object detection, distance calculation and clustering algorithms. Object detection is done by background subtraction. During it, the shadows are removed as well. Clustering is done using Unsupervised Machine Learning. The thesis also contains the analisys of the given domain, the design and implementation of the system which realizes the proposed solution. A simple model of human-robotic interaction is presented. The first chapter is concerned with defining the scope of the project, analysis of the domain and similar works. The second chapter describes in more details the mathematical algorithms and methods used during implementation. The third chapter presents the structural, behavioral and interactional design of the system. The used API-s and the implementation follows. An economical insight of the project is presented later. The thesis ends with conclusions and additional material like UML diagrams and relevant source code. This document is for readers with technical background, engineers, IT students and programmers. 