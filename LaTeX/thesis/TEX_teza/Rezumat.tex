\section*{Rezumat}

Teza \textbf{Gruparea Obiectelor Folosind Robotul NAO} prezentată de către Maxim Chetrușca ca proiect de licență a fost efectuată la Universitatea Tehnică Cehă din Praga, este scrisă în limba engleză și conține 90 de pagini, 26 de figuri, 7 tabele, 13 secvențe de cod și 20 de referințe. Teza constă din introducere, patru capitole, concluzii și anexe. Anexele mai conțin 8 figuri adiționale, un tabel și 7 secvențe de cod.

Această teză este dedicată studiului roboticii, domeniului învățării automate, interacțiunii om-robot și procesării imaginilor. Scopul acestei teze este de a elabora un sistem pe robot care l-ar face capabil să grupeze un set de obiecte în mai multe grupe, independent de tipul obiectelor sau numărul de grupe. 

Robotica devine o ramură tehnologică importantă, nu doar pentru cercetări dar și ca următoarea tendință în produsele comerciale. Marea parte a sarcinilor realizate de roboți sunt legate de interacțiunea cu obiectele și interacțiunea om-robot. Prima implică rezolvarea problemei de identificare a obiectelor în spațiu și capacitatea robotului de a le apuca. A doua implică recunoașterea vocii și reproducerea ei. Vederea este realizată prin procesarea imaginilor. Pozițiile relative a obiectelor față de robot sunt calculate pentru ca robotul să aibă o ``închipuire'' a spațiului. Această teză descrie un model de interacțiune în care un robot are de grupat un set de obiecte. Sarcina dată este îndeplinită utilizînd robotul umanoid NAO, cu ajutorul procesării imaginilor prin OpenCV și algoritmului de clusterizare ``K-means''. 

Teza conține descrierile ale algoritmilor de detectare a obiectelor, de calculare a distanței și de clusterizare. Detectarea obiectelor este realizată prin eliminarea fundalului. În decursul acestui proces se elimină și umbrele din imagine. Clusterizarea este realizată de învățarea automată nesupravegheată (Unsupervised Machine Learning). Teza mai conține și analiza domeniului dat, proiectatea și implementarea sistemului ce realizează soluția propusă. Un model simplu de interacțiune om-robot este prezentat. Primul capitol definește scopul proiectului, prezintă o analiză a domeniului și a lucrărilor asemănătoare. Al doilea capitol descrie mai detaliat algoritmii matematici și metodele numerice utilizate pe parcursul implementării. Al treilea capitol prezintă proiectarea sistemului, structura lui, partea comportamentală și interacțiunea componentelor din interiorul său. Urmează API-ul utilizat și implementarea sistemului. Următorul capitol prezintă analiza economică a proiectului. Teza finisează cu concluzii și materialele adiționale cum ar fi diagrame UML și secvențe relevante ale codului sursă. Acest document este destinat cititorilor din domeniul tehnic, inginerilor, studenților din TI și programatorilor.